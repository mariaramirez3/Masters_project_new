\documentclass{article}
\usepackage{graphicx} % Required for inserting images

\title{File tracker- Masters Project}
\author{Maria Ramírez Rodríguez}
\date{February-May 2025}

\begin{document}

\maketitle



\section{Test models}

\subsection{1NN}
FRG code (no symmetries applied) to a 1NN 2D hubbard model. Includes  a brief analysis (k1k2)  on the number of kn and knf points to use for each iteration. 

\subsubsection{Umu}


\begin{table}[h]
    \centering
    \begin{tabular}{|c|c|c|c|}
        \hline
        U & $\frac{\mu}{U}$ &  nk & nkf\\ \hline
        0-2 (steps of 0.5)  &  0.5-2 (steps of 0.5)  & 40   & 15   \\ \hline
    \end{tabular}
    \caption{Umu model}
    \label{tab:example_table}
\end{table}




\subsubsection{Umu2}
\begin{table}[h]
    \centering
    \begin{tabular}{|c|c|c|c|}
        \hline
        U & $\mu$ &  nk & nkf\\ \hline
        0-8 (steps of 0.1)  &  0-1 (steps of 0.25 (exluding 0.75))  & 40   & 10   \\ \hline
    \end{tabular}
    \caption{Umu2 model}
    \label{tab:example_table}
\end{table}


\subsubsection{Umu3CDW}
Exploring what appeared to be a CDW region, when in reality was a FL. 

\subsubsection{Umu4}

\begin{table}[h]
    \centering
    \begin{tabular}{|c|c|c|c|}
        \hline
        U & $\mu$ &  nk & nkf\\ \hline
        0-8 (steps of 0.25)  &  0-1 (steps of 0.1)  & 40   & 10   \\ \hline
    \end{tabular}
    \caption{Umu4 model}
    \label{tab:example_table}
\end{table}

\subsubsection{Umu5}

For comparison with 1.2.1 and 1.2.2

\begin{table}[h]
    \centering
    \begin{tabular}{|c|c|c|c|}
        \hline
        U & $\mu$ &  nk & nkf\\ \hline
        0-8 (steps of 0.25)  &  0-1 (steps of 0.1)  & 20   & 5   \\ \hline
    \end{tabular}
    \caption{Umu5 model}
    \label{tab:example_table}
\end{table}


\subsection{2NN}

\subsubsection{2NN-without editing}

DISCARD. Note that this model shows the results for a 2NN TBM but without the appropriate .c file

\begin{table}[h]
    \centering
    \begin{tabular}{|c|c|c|c|c|}
        \hline
        U & $\mu$ &  nk & nkf & computer \\ \hline
        0-8 (steps of 0.25)  &  0-1 (steps of 0.1)  & 20   & 15 & 0  \\ \hline
    \end{tabular}
    \caption{Umu4 model}
    \label{tab:example_table}
\end{table}


\subsubsection{2NNa }

Same model as 2NN, only subtetly is that the C file has been edited accordingly to match the two atoms per unit cell. 
 (paws1)
\begin{table}[h]
    \centering
    \begin{tabular}{|c|c|c|c|}
        \hline
        U & $\mu$ &  nk & nkf\\ \hline
        0-8 (steps of 0.25)  &  0-1 (steps of 0.1)  & 20   & 5  \\ \hline
    \end{tabular}
    \caption{Umu4 model}
    \label{tab:example_table}
\end{table}


\subsubsection{2NNb }

Same model as 2NNb, only subtetly is that the C file has been edited accordingly to ensure that the positions of the two atoms lie within one unit cell. 

\begin{table}[h]
    \centering
    \begin{tabular}{|c|c|c|c|}
        \hline
        U & $\mu$ &  nk & nkf\\ \hline
        0-8 (steps of 0.25)  &  0-1 (steps of 0.1)  & 20   & 5  \\ \hline
    \end{tabular}
    \caption{Umu4 model}
    \label{tab:example_table}
\end{table}




\end{document}