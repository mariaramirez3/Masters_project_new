\documentclass[12pt]{article} 
\usepackage{graphicx} % Required for inserting images
\usepackage{amsmath}
\usepackage{subcaption} 
\usepackage[scale=0.8]{geometry}
\usepackage{cite}

\title{Investigating unconventional superconductivity in the 2D Hubbard-Kanamori model using Functional Renomarlization Group (FRG)}
\author{210003218}
\date{February-May 2025}

\begin{document}
\maketitle
\tableofcontents 


\section{Abstract}

\section{Introduction}

\section{Theoretical Background}

\subsection{Unconventional superconductivity}

For many years after the discovery of superconductivity in 1911\cite{onnes1911superconductivity} physicists
were convinced that BCS-Eliashberg-electron-phonon theory \cite{schrieffer2018theory} provided a complete explanation of the pairing mechanism in all superconducting materials. 
However, in 1986, the discovery of the first heavy-fermion superconductor\cite{bednorz1986possible} resulted in the emergence of a whole new class of materials: Unconventional Superconductors. 
These are condensates of cooper pairs formed by a \textbf{different} pairing mechanism than the electron-phonon coupling predicted by BCS theory\cite{hirsch2015superconducting}. 

\medskip
\noindent Most generally, the Hamiltonian for a superconducting state can be described as follows:

\begin{equation}\label{General Hamiltonian}
    \hat{H} = \hat{H}^0 + \hat{H}^{cp}
\end{equation}

\noindent where $\hat{H}^{cp}$ describes the pairing interation that leads to the formation of a Cooper pair and is given by:

\begin{equation}\label{Hcp}
    \hat{H}^{cp} = \sum_{k,k'} \Gamma(k, k') c^{\dagger}_{k, \uparrow}  c^{\dagger}_{k', \downarrow} c_{k', \uparrow}c_{-k, \downarrow}
\end{equation}

\noindent For many unconventional superconductors, the form of the effective pairing interaction $\Gamma(k,k')$ has remained as an unanswered question for decades.

\subsubsection{Spin-fluctuation mediated superconductivity}

One emerging theory for some unconventional superconductors such Iron-based or heavy-fermion compounds 
is that underlying pairing mechanism is driven by spin fluctuations \cite{moriya2000spin}. This section discusses how to
model spin-fluctuation mediated superconductivity for the fluctuation-exchange approximation(FLEX) \cite{esirgen1997fluctuation}. \par
\medskip
\noindent In such cases, the effective pairing interaction $\Gamma(k,k')$ is given by \footnote{Note that this form of pairing interaction 
assumes that the ratio between 
the fluctuation frequence $w_f$ and the Fermi energy is small.}:

\begin{equation}\label{Pairing interaction SF}
    \Gamma(k,k') = \frac{3}{2} U^2 \chi^S(k-k') -\frac{1}{2}U^2 \chi^C(k-k') + U
\end{equation} 

\noindent This equation is taken from Ref.\cite{migdal1958interaction}. Here, U is the on-site Coulumb repulsion and $\chi^S$, $\chi^C$ are the interacting spin-susceptibilities in the Charge (C) and Spin(S) channel respectively. Their form is given below.

\begin{equation}
    \chi^S(q) = \frac{\chi^0(q)}{1 - U \chi^0 (q)}
\end{equation}

\begin{equation}
    \chi^C(q) = \frac{\chi^0(q)}{1 + U \chi^0 (q)}
\end{equation}

\noindent These interacting spin susceptibilities are expressed in terms of the non-interacting dynamic spin susceptibility ($\chi_{ps}^0$) which is stated below without formal proof \cite{moriya2000spin}. 

\begin{equation}\label{chi 0}
    \chi_{ps}^0(q, i \omega) = -\sum_{k} \int_{0}^{\beta} d\tau G^0_{ps}(k+q \tau) G^0_{sp}(k, -\tau)e^{i\omega \tau}
\end{equation} 

\noindent The calculations presented in this project are carried out in the spin-fluctuation framework. 
Whilst this theory has managed to succesfully capture key features in phase diagrams of unconvential superconductors, it also has its limitations.
The most relevant example is that of the cuprate phase diagram.
Spin-fluctuation theory is able to capture the superconducting dome and the correct order parameter \cite{moriya2006developments, scalapino1995case} 
but fails
to describe the characteristic pseudo-gap\cite{timusk1999pseudogap}. 

\subsection{Hubbard-Kanamori Model}

\subsubsection{Tight Binding Models}

The Tight Binding Model is a central element of condensed matter physics \eqref{TBM}. In this model, electrons are bound in orbitals (called sites) around the lattice ions.
Due to the overlap between the quantum mechanical wavefunctions that describe these sites, electrons are allowed to 'hop' to neighbouring sites. The probability that this hopping process will occur is given by a tunnelling amplitude, which can be calculated using a hopping integral. \par
\medskip
\noindent This work is carried out in the tight-binding model framework, where the magnitude of the tunnelling amplitudes are at first treated as free parameters. 
Here, we build from a simple 2D nearest-neighbour hoppping Hubbard model (Section.~\ref{subsec: HubbardModel}) and investigate the effect of introducing and varying the strength of the next-nearest neighbour hopping amplitude (See Fig. \ref{fig:2D Hubbard model}).
We extend these models further by considering the two-orbital per site case. 



\begin{equation} \label{TBM}
    \hat{H}_{TB}(\b{R}) = \sum_{ij\sigma} t_{ij}(\hat{c}_{i\sigma}^{\dagger}\hat{c}_{j \sigma} + h.c)
\end{equation}


\begin{figure}[htbp]  % Placement: Here, Top, Bottom, Page
    \centering
    \includegraphics[width=0.9\textwidth]{2Dhubbardmodel.png}  % Adjust width as needed
    \caption{\textbf{Two-Dimensional Tight-Binding Models:} Three pannels showing the tight binding models for the 1NN, 1NNN and 1NN2 models discussed in Section~\ref{subsec:1NNModel},~\ref{subsec:1NNNModel} and~\ref{subsec:1NN2Model} respectively. Fig a) shows the Nearest-neighbour hopping case, where \textit{t} depicts the hopping amplitude between the neighbouring sites. Fig b) shows the inclusion of the next-nearest neighbour hopping- magnitude given by \textit{t'}.
    Fig c) Shows the extension to the two-orbital case, depicting  same orbital ($t_{orb1}$, $t_{orb2}$) and different orbital ($t_{orb1,2}$) nearest-neighbour hopping. Note that this is just a pictorial representation of the orbitals, and that it does not correspond to a particular choice of orbitals or their real space projection. }
    \label{fig:2D Hubbard model}
\end{figure}

\newpage

\subsubsection{Hubbard Model}
\label{subsec: HubbardModel}

The  tight binding model as defined above fails to account for any interactions between neighbouring electrons. This motivates the extension of this model to the Hubbard model \eqref{t Hubbard model}, which includes the (onsite) Coulomb repulsion between electrons. Despite its simple form, this model can describe very rich physical phenomena.
In particular, it becomes very interesting to study when U and t are of comparable order, since it highlights the competing phenomena that takes place in correlated systems. 
The 2D Hubbard model remains unsolved to date, but is able to predict all sorts of correlated phases: it describes metals, insulators, superconductors and other exotic phases\cite{white1989numerical,hirsch1985two, anderson1990luttinger,sun2011nearly}. 
This model has been widely studied since it resembles the structure of the cuprate high-temperature superconductors \cite{dagotto1994correlated}. 




\begin{equation}\label{t Hubbard model}
    \hat{H} = \sum_{ij\sigma} -t_{ij}(\hat{c}_{i\sigma}^{\dagger}\hat{c}_{j \sigma} + h.c) 
    + U \sum_{i} \hat{n}_{i \uparrow} \hat{n}_{i \downarrow}
\end{equation}





\subsubsection{Hubbard-Kanamori Model}
In the case of materials with a multi-band and/or multi-orbital nature, the Hubbard model is not sufficient to capture all of the physical phenomena. This motivates the extension of the Hubbard Model to the Hubbard-Kanamori model\cite{sherman2020hubbard} by including a Hund's coupling term.

\begin{equation} \label{Hubbard-Kanamori Model}
    H_{int} = U \sum_{is}n_{i,s\uparrow}n_{i,s\downarrow} + \frac{V}{2} \sum_{i,s,t \neq s} n_{is}n_{it} -\frac{J}{2} \sum_{i,s,t \neq s} \vec{S}_{is} \cdot \vec{S}_{it} 
    + \frac{J'}{2} \sum_{i,s,t \neq s} \sum_{\sigma} c_{is\sigma}^{\dagger}c_{is\bar{\sigma}}^{\dagger}c_{it\bar{\sigma}}c_{it\sigma}
\end{equation}

\noindent Here, $U$ and $V$ represent the electronic interactions in the same and different orbitals respectively. For generality, the intraorbital exchange $J$ and the 'pair hopping' term $J'$ following from Hund's rule coupling have been separated.  
Note that this Hamiltonian is relevant for the later section of this project, where the model is extended to a two-orbital, two-dimensional Hubbard Model.

\subsection{Theoretical Background in FRG}

Solving the Hubbard-Kanamori Hamiltonian is rather challenging, which is why we resort to numerical techniques such as FRG to do so. FRG falls into the category of many other weak-coupling techniques (such as Mean field-theory\cite{kadanoff2009more}, pertubation theory\cite{nagaosa2013quantum}, Density Functional theory\cite{kohn1965self} or Random Phase approximation\cite{bohm1951collective}).
In these theories, interactions between electrons are considered to be weak. This allows one to effectively model the electrons in the system as free particles 
and treat their interactions as a pertubation. In the non-interacting limit, the method is therefore exact. Beyond this limit, it is controlled by the ratio between the interaction
strength and the bandwidth of the system. 

\medskip

\noindent In this section the theoretical framework in which FRG calculations are performed is outlined. The central element of FRG is a flow equation that describes the evolution of the effective action of the system with respect to a scalar/flow parameter Lambda $\Lambda$ \textit{(See section ~\ref{subsubsec:Flow Equation})}.
The flow equation can be solved exactly for a limited number of systems, so for most scenarios an approximation has to be made in order to reach a solution in a reasonable computational time. More details of how this approximation is performed and the limitations it presents can be found in Section ~\ref{subsubsec:Truncation scheme}. For the systems explored in this project only two-particle
interactions are considered and any higher order terms are neglected.
This allows for the effective action of the system to be separated into three terms, which correspond to three physical channels: Superconductivity, Spin-Density and Charge-Density Waves.
The calculation is then performed to determine the "winning channel" which will correspond to the respective physical phase that the model exhibits. This rather "hand-wavy" overview of FRG is represented in a flowchart in Fig.\ref{fig:FRGflowdiagram}.
For a more rigorous explanation the reader is referred to the sections below. 

\begin{figure}[htbp]  % Placement: Here, Top, Bottom, Page
    \centering
    \includegraphics[width=0.9\textwidth]{FRGflowdiagram.png}  % Adjust width as needed
    \caption{\textbf{FRG Flowchart:} Schematic diagram outlining the TU$^2$FRG calculation steps. Starting from the flow equation and applying the truncation scheme in order to decouple the action into three "physical" terms. 
    The flow equation can then be solved by calculating each channel separatedly and taking the dominating phase to be the channel that diverges.}
    \label{fig:FRGflowdiagram}
\end{figure}

\subsubsection{Flow equation}
\label{subsubsec:Flow Equation}

In this section the derivation of the flow equation is outlined. For such, it is assumed that the reader has grasped a strong understanding in Quantum Field theory and many-body physics.
If interested in the finer details of the derivation, the reader is refferred to \cite{metzner2012functional}. \par
\medskip
\noindent The central elements of statitiscal physics are the partition function, the canonical potential and its Legendre transformations. 
These are such powerful physical quantities that one can derive all physical observables from them.
For quantum many-body problems, one works instead with partition functional, defined as follows in Eq.\ref{partition functional}:

\begin{equation}\label{partition functional}
    \mathcal{Z}[\bar{\eta}, \eta] = \int \mathcal{D} \bar{\psi} \mathcal{D}\psi e^{\mathcal{S}[\bar{\psi}, \psi]}e^{(\bar{\eta}, \psi)+(\eta, \bar{\psi})}
\end{equation}

\noindent In the case of fermionic systems, the action in the exponent of Eq.\ref{partition functional} takes the form shown below.
\begin{equation} \label{action}
    \mathcal{S}[\psi, \bar{\psi}] = -(\bar{\psi}, G_0^{-1} \psi) + V[\psi, \bar{\psi}]
\end{equation}

\noindent Here, $V[\psi, \bar{\psi}]$ is an arbitrary many-body interaction and $G_0$ represents the propagator of the non-interacting system. This equation contains the shorthand notation $(...)$, which represents the sum $\sum_x \bar{\psi}(x)(G_0^{-1}\psi)(x), (G_0^{-1}\psi)(x) = \sum_{x’}G_0^{-1}(x,x’)\psi(x’)$. In this sum, the Grassman field index $x$ represents all the quantum numbers of the single-particle basis and imaginary time.\par
\medskip

\noindent Note that in the limiting case where $V=0$, the path integral in Eq.\ref{partition functional} is exactly solveable. 
However, once electronic correlations are included, the picture becomes more complicated. 
The main idea behind FRG is to introduce a cut-off in the non interacting Green's function ($G_0 \rightarrow G_0^{\lambda} = f(\lambda)G_0$). 
This cutoff is then interpolated between the solveable intital state and the full path integral solution by susbsequently including electronic interactions. For a spin-independant system this would transform the 
bare propagator as shown in Equations (\ref{propagator transform 1}) and (\ref{propagator transform 2}). 

\begin{equation} \label{propagator transform 1}
    G_0(k_0, \textbf{k}) \rightarrow G_0^{\lambda}(k_0, \textbf{k})
\end{equation}


\begin{equation} \label{propagator transform 2}
    \frac{1}{ik_0 - \xi_{\textbf{k}}} \rightarrow \frac{\theta^{\textbf{k}}}{ik_0 - \xi_{\textbf{k}}}
\end{equation}

\noindent where $\theta^{\lambda}(\textbf{k})$ is defined, for example, as follows:

\begin{equation} \label{theta def}
    \theta^{\lambda}(\textbf{k}) = \Theta(|\xi_{\textbf{k}}| - \lambda)    
\end{equation}

\noindent With this cutoff scheme, the calculation then excludes points close to the Fermi Surface as shown in Figure (FIGURE). \par
\begin{figure}[htbp]  % Placement: Here, Top, Bottom, Page
    \centering
    \includegraphics[width=0.35\textwidth]{Truncation.png}  % Adjust width as needed
    \caption{\textbf{Cut-off scheme example:} Momentum space region(shaded in grey) around the Fermi-surface(red) that is excluded 
    by a momentum cut-off for a 2D square lattice with a lattice constant of 1\AA. Taken from \cite {metzner2012functional}}.
    \label{fig:Truncation}
\end{figure}

\medskip
\noindent In the following steps the derivation will proceed in the framework of the  so-called "effective action" ($\mathcal{T}[\psi, \bar{\psi}]$).
This is the Legendre transformation of the Greens function functional ($\mathcal{G}[\eta, \bar{\eta}]$), defined below in Equations (\ref{Greens function functional}, \ref{G term in effective action}) and (\ref{Effective action}) respectively. 
\textit{(For convinience and for reasons that
are beyond the scope of this project it is more convinient to work with the effective action than it is to do so with the partition functional.) \footnote{If interested in why see REFERENCE}}

\begin{equation} \label{Greens function functional}
    \mathcal{G}[\eta, \bar{\eta}] = -ln(\mathcal{Z}[\eta, \bar{\eta}])
\end{equation}

\begin{equation}\label{G term in effective action}
    \mathcal{G}[\eta, \bar{\eta}] = 
    -ln \int{\mathcal{D}\psi \mathcal{D} \bar{\psi}e^{-\mathcal{S}[\psi, \bar{\psi}]}e^{(\bar{\eta}, \psi) +(\bar{\psi}, \eta)}}
\end{equation}

\begin{equation} \label{Effective action}
    \mathcal{T}[\psi, \bar{\psi}] = (\bar{\eta},\psi) + (\bar{\psi},\eta) + \mathcal{G}[\eta, \bar{\eta}]
\end{equation}


\noindent The next step is to introduce a scalar flow parameter $\lambda$ into the generating functionals defined above. This is done in the same manner as is outlined in the example shown in Equations (\ref{propagator transform 1}, \ref{propagator transform 2}). 
But more generally, has to be performed such that the generators recover their orginal structure at $\lambda = 0 $.
After a series of algebraic manipulations, which are ommitted here but can be found in \cite{metzner2012functional}, one arrives at the exact functional flow equation for the effective action:


\begin{equation} \label{eq:ExactFunctionalFlowEquation}
    \frac{d}{d\Lambda} \mathcal{T}^{\Lambda}[\psi, \bar{\psi}] = (\bar{\psi}, \dot{Q}_0^{\Lambda} \psi) - \frac{1}{2} \text{tr} \big( \dot{Q}_0^{\Lambda} (\boldsymbol{\Gamma}^{(2)\Lambda}[\psi, \bar{\psi}])^{-1} \big).
\end{equation}

\noindent Where $\Gamma^{(2)\lambda}[\psi, \bar{\psi}]$ and  $\b{Q}_0^{\Lambda}$ are given by equations (\ref{Gamma term}) and (\ref{Q0 term }) respectively.



\begin{equation}\label{Gamma term}
\Gamma^{(2)\lambda}[\psi, \bar{\psi}] = 
\begin{bmatrix}
\bar{\delta} \delta \Gamma[\psi, \bar{\psi}](x',x) & \bar{\delta} \bar{\delta} \Gamma[\psi, \bar{\psi}](x',x) \\
\delta \delta \Gamma[\psi, \bar{\psi}](x',x)  & \delta \bar{\delta} \Gamma[\psi, \bar{\psi}](x',x)
\end{bmatrix}
\end{equation}



\begin{equation}\label{Q0 term }
\b{Q}_0^{\Lambda} =
\begin{bmatrix}
Q_0^{\Lambda} & 0 \\
0 & -Q_0^{\Lambda t}
\end{bmatrix}
= diag(Q_0^{\Lambda}, - Q_0^{\Lambda t}),
\end{equation}

\noindent This equation, as mentioned previously, is the central element of FRG and the sections below outline how to solve it.


\subsubsection{Truncation scheme (TU$^2$FRG)}
\label{subsubsec:Truncation scheme}

The flow equation derived above can be solved for a limited amount of systems. However, in most cases, the memory demand for the 
FRG calculations is high.  In order to tackle this issue, the truncated unity approximation (TU$^2$FRG) was introduced in 2020\cite{eckhardt2020truncated}. 
The main idea behind this scheme is to find a new basis that, with a controlled loss of accuracy, can represent all 
of the required elements in a compressed way. It can be shown\cite{lichtenstein2018functional}, that such a basis can be constructed and is 
well defined in the case where the calculation is constrained to terms U$^{2}$ \textit{(two-particle interactions)}.
Partitions of unity \footnote{A detailed explanation of what these are can be found in \cite{lichtenstein2018functional}} are then introduced into a specific part of the flow equation. This reduces an otherwise computationally expensive nested integral to a matrix product. 
Details of how this truncation is incorporated are ommitted but the reader is referred to Appendix. (REFERENCE appendix).\par
\medskip

\noindent Whilst the truncated scheme presents advantages in computational efficiency, particularly for models with broken translational symmetry, it also has its limitations. 
TU$^2$FRG relies on short-range interactions, thus struggling to capture strongly correlated phases. 
This is particularly relevant for the study of the 2D Hubbard model. TU$^2$FRG will not be able to capture
the characteristic Mott insulating phase of the Cuprate phase diagrams\cite{imada1998metal}, which limits how well the results presented in this project can be directly compared with existing literature. 
More importantly, the truncation scheme has a direct consquence on the accuracy of the predicted phase transition temperature ($T_c$), it is able to correctly capture
the trends in $T_c$ but the values predicted are much higher than what is sensible to expect in real materials.
Nevertheless, TU$^2$FRG manages to successfully capture the competition between Magnetic and Superconducting instabilities, which is one of the main focus of the results presented here. 
\subsubsection{Decoupling of flow equation}
After constraining ourselves to the case of two-particle interactions in the framework of translationally invariant systems, one can decouple  the evolution of the two-particle coupling as a function of the flow parameter $\lambda$ into three channels:

\begin{equation} \label{V decoupling}
    V(k1,k2,k3)= V_{k_1, k_2, k_3}^{(0)} - \phi^{P}_{k_1 +k_2, \frac{k_1 - k_2}{2}, \frac{k_4-k_3}{2}} + \phi^{C}_{k_1 - k_3, \frac{k_1 +k_3}{2}, \frac{k_2+k_4}{2}} +\phi^{D}_{k_3- k_2, \frac{k_1 + k_4}{2}, \frac{k_2+k_3}{2}}
\end{equation}

\noindent Here, the three channels correspond to a particle-particle, crossed particle-hole and (three) direct particle-hole terms given explicitly below in terms of the respective effective actions. They represent all possible ways in which the two particle interactions can occur in the correlated system. The particle-particle (P), cross-particle-hole (C) and direct-particle-hole (D) channels
correspond to the Superconducting, Charge and Magnetic phases respectively. 

\begin{equation}
    \dot{\phi}^{P}_{k_1 +k_2, \frac{k_1 - k_2}{2}, \frac{k_4-k_3}{2}} = - \mathcal{T}_{pp}(k1,k2,k3)
\end{equation}


\begin{equation}
    \dot{\phi}^{C}_{k_1 - k_3, \frac{k_1 +k_3}{2}, \frac{k_2+k_4}{2}} = - \mathcal{T}_{cr-ph}(k1,k2,k3)
\end{equation}

\begin{equation}
    \dot{\phi}^{D}_{k_3- k_2, \frac{k_1 + k_4}{2}, \frac{k_2+k_3}{2}} = - \mathcal{T}_{d-ph}(k1,k2,k3)
\end{equation}

\noindent This effectively allows one to treat the effective action as a separable object:

\begin{equation}
    \mathcal{T}[\psi, \bar{\psi}] = \mathcal{T}_{pp}[\psi, \bar{\psi}] + \mathcal{T}_{ph}[\psi, \bar{\psi}] + \mathcal{T}_{cph}[\psi, \bar{\psi}]
\end{equation}

\medskip





\subsubsection{Instability calculation}

The procedure from the decoupled  effective action and flow equation is outlined here. For more details the reader is refferred to\cite{profe2023functional}. 
A FRG calculation will return the self-energy and two-particle vertex \footnote{An exact definition of these can be found in \cite{metzner2012functional}, but is omitted here.}. These quantities are
not directly experimentally accessible, so a series of post processing steps have to be carried out. The first step is to find out which of the three channels diverge. 
This gives the diverging susceptibility. Mean-Field analysis is then carried out at the critical scale in order to obtain
the ordering symmetry and a linearised gap equation. The later allows one to calculate the Superconducting gap and order parameter. 



\section{Computational Methods}

This section outlines how the concepts presented above link together and are implemented for the purpose of this project. 
The general aim is to solve the 2D Hubbard-(Kanamori) model. This is achieved computationally, using the divERGe package (See section ~\ref{subsec:diverge}) which implements the TU$^2$FRG
formalism discussed in the section above. All of the calculations are carried out under the assumption that the superconductivity 
is spin-fluctuation mediated. This section outlines the methods followed to calculate the required Tight-Binding Models for each of the systems, the utilisation of 
Diverge to solve them and some of the utilised post-processing techniques. 

\subsection{Tight-Binding Models}

This project discusses three models: the 1NN, 1NNN and 1NN2 model (\textit{See Fig.\ref{fig:2D Hubbard model}}). For the first two, the hopping parameters in the tight-binding model are treated as free parameters. 
For the later, the hopping parameters are determined using the method described below. \par
\medskip
\noindent Constrained to the case of a single layer, the 1NN model is extended to a multi-orbital system by
including two-orbitals per site and investigating the effect of the interorbital hopping. To allow for comparison with
existing literature\cite{sakakibara2024possible}, a $d_{x^2-y^2}$ and $d_{3z^2 -r^2}$ orbital are included. 
Their respective hopping parameters are determined using the  table of interatomic matrix elements calculated by J.C. Slater and G.F. Koster\cite{slater1954simplified}. 
In this calculation, the values of the $\sigma,  \pi, \delta$ bond strength  are approximated to  $\approx  1, 0.5, 0.05eV$ respectively in order to capture their relative values\cite{blanksby2003bond, mcgrady2015introduction, krapp2008strength}.
A table summarising estimated parameters for the models  discussed in Section~\ref{subsec:1NN2Model} is shown below.

\begin{table}[h]
    \centering
    \begin{tabular}{|c|c|c|c|c|c|}
        \hline
       Model &$t^{[1,0,0]}_{3z^2-r^2}$  &$t^{[0,1,0]}_{3z^2-r^2}$  &  $t_{x^2 - y^2}$ &  $t^{[1,0,0]}_{x^2 - y^2 -3z^2-r^2} $ & $t^{[0,1,0]}_{x^2 - y^2 -3z^2-r^2} $ \\
        \hline
        & -0.781 & -0.719  &  -0.375 & -0.402 & -0.310\\
        \hline
        1NN2MN & on &  on  & on  & off & off  \\
        \hline
        1NN2MY & on &  on  & on  & on & on \\
          
        \hline
    \end{tabular}
    \caption{\textbf{Nearest neighbour hopping parameters for 2D  two-orbital Hubbard models.} First row shows the calculated hopping parameters. Other rows show which hopping parameters were included in each of the two models.}
    \label{tab:2D2orbparams }
\end{table}

\subsection{divERGe}
\label{subsec:diverge}

divERGe is an open-source, high-performance, C/C++/Python library that includes a truncated unity FRG (TU$^2$FRG) computational backend\cite{profe2024diverge}. 
Under the approximations outlined in Section\ref{subsubsec:Truncation scheme}, the flow equations
are integrated fromhigh scales($\lambda = \inf$) to low scales ($\lambda = 0$) by numerically going from $\lambda$ to $\lambda + d\lambda$.
This integration is repeated until a phase transition (divergence of one of the calculated channels) occurs or a minimal $\lambda$ is reached. 
In the later case, the system is assumed to be in a Fermi-Liquid state. 
All calculations have been carried out in the HPC cluster at the university of St Andrews. 


 

\subsection{Convergence of Calculation}
\label{subsec:convergence}
In the truncated-unity approximation there are several convergence tests that one needs to carry out in order to verify
that the calculations are accurate. These include the form factor convergence and the number of k points.

\subsubsection{Form factor convergence}

The set of orthogonal basis functions ($f_m$) used to describe the Truncated space\textit{(See section \ref{subsubsec:Truncation scheme})}
in momentum representation is called the \textbf{form factors}. In the case of the square lattice, 
these take the form of delta functions in real space. The form factors are arranged as circles with increasing radii around the origin. This effectively leads to a "bond-like" representation where the form-factor number
essentially determines how many of the neighbouring bonds are accounted for in a calculation for each point. This is depicted in Fig. . For mathematical rigour on the definition of the form factor, the reader is referred to \cite{lichtenstein2018functional}. \par

\medskip


\noindent The divERGe package allows the user to modify the number of form-factor shells accordingly. It boils down to a trade 
between computational accuracy and expensiveness. 
For the results in this project, the form factor value was set at 4\AA\footnote{This is equivalent to a number of form factor shells of 4 since the lattice spacing of the models here is set to 1\AA.}. The convergence of 
the calculations was tested accordingly for a range of points in the phase diagram (an example is shown in Fig.\ref{fig:Formfactorconvergence} ). Moreover, this is in agreement with values used in previous literature\cite{lichtenstein2018functional}. 

\begin{figure}[htbp]  % Placement: Here, Top, Bottom, Page
    \centering
    \includegraphics[width=0.6\textwidth]{convergence.png}  % Adjust width as needed
    \caption{\textbf{Convergence testing:} Fig a) Transition temperature as a function of form factor for the 1NN model, U = 5.00, $\mu$ = 0.20eV. Fig b) Time taken for calculation
    as a function of form factor.   }
    \label{fig:Formfactorconvergence}
\end{figure}

\newpage





\subsubsection{Number of k points convergence }

When integrating the flow equation, there are two parameters that can be further tuned to ensure convergence.
Those are $n_k$ and $n_{k_f}$ and  loosely speaking specify the number of k points used to carry out the nested integrations. 
In particular, $n_k$ refers to the number of k points used for the general momentum integral, and $n_{k_f}$ for 
the additional sum around each k-point. The calculations performed here were carried out with an integration 
grid of 20x5 ($n_k$ x $n_{k_f}$) points. The choice of parameters ensured that the calculations had converged appropriatedly 
and resulted in a computational time of $\approx 180s$ per point. 

\subsection{Calculation of Susceptibilities}

\subsubsection{Nesting vectors}

\subsubsection{Superconducting order parameters}

\section{Results and discussion}

\subsection{1NN Model}
\label{subsec:1NNModel}

We define the 1NN model as the 2D Hubbard model with a single orbital per site,  allowing hopping between nearest-neighbour sites \textit{(a diagramatic representation can be found in Fig.\ref{fig:2D Hubbard model}.a)}. 
As outlined in Section \ref{subsec: HubbardModel}, solving this model near half-filling becomes a hard endeavour. 
This section explores the solution to the 1NN Model Hamiltonian using  two-particle-interaction-truncated FRG.
Here, a phase diagram in terms of the On-site Coulumb repulsion $U$ and chemical
potential $\mu$ is presented and discussed in detail. The calculations were carried out using a 20x5 $n_k$x$n_{kf}$ grid and a form factor
of 4\AA, with a  nearest-neighbour hopping parameter of 1eV.  The choice of convergence parameters was determined by the methods discussed in Section \ref{subsec:convergence}. 
Results are presented for Coulumb repulsion values  ranging from 1-20 eV and chemical potential
values spanning the entire energetic bandwith of the model (from -4eV to 4eV). 
Whilst there exists previous work on the 2D Hubbard Model using FRG\cite{beyer2023rashba,hille2020quantitative,vilardi2020dynamical} most of it focuses on specific areas of the phase diagram and there is very little focus on the effect of varying the 
onsite Coulumb repulsion. Therefore, the work presented here covers a much wider range of parameters than what has previously been explored. 
\medskip




\noindent There are several caveats to the results presented here. In real materials,  the chemical potential is an easily tuneable parameter due to its strong connection to the electronic doping of the system.
However, controlling the magnitude of the Coulumb repulsion between electrons is by no means straightforward. 
Moreover, some of the interesting features discussed in this report lie outside of the physical regime. (ie- in our FRG calculation a weak-coupling limit was 
assumed and therefore any Coulumb repulsion values above the magnitude of the  bandwidth of the material (8eV) can be considered to be unphysical in this limit.)
Therefore, the analysis performed in this project does not neccesarily provide a route to enhance the superconductivity in real materials that resemble the models investigated. Instead, the main aim is to understand how varying hopping parameters affects the correlated phases observed in the model. It is also important to note, as discussed in Section \ref{subsubsec:Truncation scheme}, that any analysis on the transition temperature
of the superconducting regions must be treated qualitatively and that this work does not claim to have found Superconducting regions at temperatures of the order of thousands of Kelvin in any of the models discussed. \par 

\medskip
\noindent The complete phase diagram for the values discussed above is shown in Fig.\ref{fig:1NNpd}. As is expected
for both the Hubbard model and the Cuprates\cite{kivelson1998electronic,fradkin2015colloquium,vanhala2018dynamical}, a strong competition between Magnetism and 
Superconductivity is observed. This results in the emergence of a magnetic dome, sandwiched between two narrower d-wave superconducting regions. This magnetic dome is 
\textbf{Anti-Ferromagentically} ordered and its width increases as the magnitude of the Coulumb repulsion is increased. A greater Coulumb repulsion thus favours the magnetic instability for a larger range
of doping values. Since both the AFM and d-wave SC are driven by repulsive scattering between
($\pi$,0) and (0,$\pi$) vectors, their competition is at its largest closest to the Van-Hove singularity, where both grow and reinforce the other\cite{furukawa1998truncation,honerkamp2001temperature}. Moreover, narrow magnetic stripes appear for even
integer chemical potential values. For large values of U (greater than the material's bandwidth) there are patches of a charge density wave instability sorrounding the superconducting regions.\par

\medskip

\noindent The findings presented here are consistent with other studies of the two-dimensional Hubbard model. Dynamical Mean Field Theory(DMFT) has previously captured the idea that at weak-coupling the antiferromagnetic and d-wave superconductivity order parameters co-exist within the same solution
for a range of doping (chemical potential values) and that the sytem evolves smoothly from the AFM to the SC phase\cite{capone2006competition}. 





\begin{figure}[htbp]  % Placement: Here, Top, Bottom, Page
    \centering
    \includegraphics[width=1.25\textwidth]{1NNphasediagram.png}  % Adjust width as needed
    \caption{\textbf{Phase diagram for the 1NN model}:  (t =1eV, nkxnkf = 20x5, ff = 4\AA) as a function of On-site Coulumb Repulsion $U$ and chemical potential $\mu$. 
    Figure shows the four phases observed in the 1NN model: SC (Superconductivity), SDW (Spin-Density Wave), CDW(Charge Density Wave) and FL (Fermu-Liquid).
    Calculated points in the phase diagram are showed by the 'x' markers and a lighter-couloured background is used to depict interpolated regions between these points. }
    \label{fig:1NNpd}
\end{figure}

\newpage


\subsubsection{Superconductivity in the 1NN Model}

Recent findings have claimed that the 2D Hubbard model does not have a superconducting ground state\cite{qin2020absence}. 
Whilst their model is the same as the 1NN model explored in this section, the analysis in Ref. \cite{qin2020absence} is conducted in a 
slightly different framework than the one used in this project: using DMRG \cite{white1992density} 
at moderate-strong coupling and for values of U between 6-8eV\textit{(which they claim to be the 
regime relevant to the cuprates)}.
The lack of superconductivity in their findings is attributed to a lack of competition 
between the magnetic and superconducting phases. Here, in the weak-coupling framework both a competition between magnetism and superconductivity and 
a stabilised superconducting region is found. The superconducting order parameter is plotted on top of the Fermi-Surface for the first BZ of points in the Superconducting region (Fig.\ref{fig:1NNSC}a).
The order parameter shows antisymmetry with respect to a 90 degree rotation and therefore
the superconducting region is d-wave symmetric. This is in agreement with what is expected from the Cuprate superconductors \cite{tsuei2000pairing}.\par
\medskip
\noindent 
Moreover, after plotting the the critical temperature (Tc) as a function of chemical potential ($\mu$) for a constant Coulumb repulsion(U), one observes that
the superconducting critical temperature is maximised closest to the magnetic instability (Fig.\ref{fig:1NNSC}b). This matches existing theories that the competition between instabilities
enhaces the phase transition temperature(REFERENCE). The magnitude of the Coulumb repulsion and Tc are positively correlated, as is to be expected from the assumed of the pairing mechanism\textit{(See Eq.\ref{Pairing interaction SF})}.
It is important to note that limiting the interactions 
we consider in the FRG calculation to two-particle interactions acts as a bottle-neck for the accuracy 
of the Tc values calculated. Nevertheless, the trends in Tc remain trustworthy. 

\begin{figure}[htbp]  % Placement: Here, Top, Bottom, Page
    \centering
    \includegraphics[width=1.0\textwidth]{1NNSC.png}  % Adjust width as needed
    \caption{\textbf{Superconductivity in the 1NN model}:  
    Fig a) Superconducting order parameter projected on Fermi Surface at $\mu$ = 1.00 eV, 
    for U =10.00eV. The order parameter is antisymmetric about a 90 degree rotation and hence
    it exhibits d-wave symmetry. 
    Fig b) Transition temperature (Tc) as a function of chemical potential ($\mu$) for U =10.00eV.
    Tc is enhanced closest to the magnetic instability. 
    Although both plots are shown only specific for certain values of U and $\mu$, the results they display hold for the enterity of
    the superconducting region of the phase diagram in Fig.\ref{fig:1NNpd}.  
    }
    \label{fig:1NNSC}
\end{figure}



\subsubsection{Magnetic stripes in the 1NN Model}

\begin{figure}[htbp]  % Placement: Here, Top, Bottom, Page
    \centering
    \includegraphics[width=0.9\textwidth]{1NNstripes.png}  % Adjust width as needed
    \caption{\textbf{Magentic stripe in the 1NN model for doping at $\frac{1}{8}$ of the bandwith}: Fig a) Transition temperature as a function of Coulumb repulsion (U) along the Magnetic stripe at $\mu$ =1eV. 
       Fig b) Magnetic susceptibility along high symmetry path for U =2.00 eV. 
       Fig c) Plot of the Superconducting order parameter projected on top of the Fermi-surface of the 1NN model for U = 10.00eV and $\mu$ =1.00eV.
       Fig d) Magenetic susceptibility along high symmetry path for U= 18.00eV. Plots showing the susceptibility as a function of \b{q} for both magnetic regions. The Ferromagnetic SDW is supressed by a superconducting phase at U$\approx$ 6.00eV. At larger values of U,  the SDW phase is recovered but with an Anti-Ferromagnetic ordering instead.  }
    \label{fig:1NN_stripes}
\end{figure}


\subsection{Effect of next-nearest neighbour hopping (1NNN model)}
\label{subsec:1NNNModel}


\begin{figure}[htbp]  % Placement: Here, Top, Bottom, Page
    \centering
    \includegraphics[width=1.0\textwidth]{1NNN.png}  % Adjust width as needed
    \caption{\textbf{1NNN model:} a) Phase diagram for the 1NNN model as a function of Coulumb repulsion U and chemical potential $\mu$ (t =1eV, nkxnkf = 20x5, ff = 4\AA) for t'=0.00, 0.25, 0.50, 0.75 eV. 
    b) Transition temperature for the superconducting region.
    c) Single plot of the band structure along high-symmetry path for corresponding values of t'.
    d) Single plot of the Fermi surface for all values of t' considered.}
    
    \label{fig:1NNN}
\end{figure}

\newpage

\subsubsection{Superconductivity in the 1NNN Model}

Important to note that results shown in Fig \ref{fig:1NNNSC} are the plots for the case of U = 9.00eV and although this
value is unphysical since we constrain ourselves to the weak-coupling approximation in which we assume that U is of the orders of
magnitude of the bandwidth of the material. 


\begin{figure}[htbp]  % Placement: Here, Top, Bottom, Page
    \centering
    \includegraphics[width=0.90\textwidth]{1NNNSC_075.png}  % Adjust width as needed
    \caption{\textbf{Change in the superconducting order parameter in the t' =0.75eV 1NNN model}: Fig a) Critical temperature
    as a function of chemical potential $\mu$ for u = 9.00eV.
    The lower panel shows the magnitude of the superconducting order parameter plotted on top of the fermi surface for U= 9.00eV and $\mu$ = -2.6eV, -1.8eV and -1.4eV. These are shown in Figs b)
    c) and d) respectively.  }
    \label{fig:1NNNSC}
\end{figure}




\subsubsection{Magnetic stripes in the 1NNN model}

\begin{table}[h]
    \centering
    \begin{tabular}{|c|c|c|c|c|c|c|}
        \hline
       t'(eV) & $\mu$ (eV) & Competing with other phase?& Nesting vector & Magnetic ordering  \\
        \hline
        0.00 & 1.00 & Yes (SC) & (0,0)-($\pi$, $\pi$)&  Changing from FM-AFM\\
        \hline
        0.00 & 2.00 &  No  & (0,0)-(0, $\pi$)  & Changing from FM to Commensurate\\
        \hline
        0.25 & 2.40 &  No  & (0,0)  & FM \\
        \hline
        0.25 & 1.40 &  Yes (CDW)  & (0,0)  & FM \\
        \hline
        0.25 & -2.00 & No  & (0,0)-($\pi$, $\pi$)  & FM - AFM\\
        \hline
        0.25 & 0.60 &  No  & (0,0)  & FM \\
          
        \hline
    \end{tabular}
    \caption{\textbf{Survey of stripes in the 1NNN model.} Table summarising the location and magnetic ordering of the Stripes in the 1NNN model .}
    \label{tab:StripesSummary}
\end{table}

\subsubsection{Continous variation of next-nearest neighbour hopping}

\subsection{Effect of bi-orbital system (1NN2 model)}
\label{subsec:1NN2Model}





\subsubsection{Superconductivity in the 1NN2 model}

\subsubsection{Effect of orbital hybridisation in the 1NN2 model}



\section{Conclusion and Outlook}


\newpage

\bibliographystyle{unsrt}  % Choose a style such as plain, alpha, abbrv, etc.
\bibliography{references}

\end{document}