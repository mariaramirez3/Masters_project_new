\documentclass[12pt]{article} 
\usepackage{graphicx} % Required for inserting images
\usepackage{amsmath}
\usepackage{subcaption} 
\usepackage[scale=0.8]{geometry}
\usepackage{cite}

\title{Investigating unconventional superconductivity in the 2D Hubbard-Kanamori model using Functional Renomarlization Group (FRG)}
\author{210003218}
\date{February-May 2025}

\begin{document}
\maketitle
\tableofcontents 


\section{Abstract}

\section{Introduction}

\section{Theoretical Background}

\subsection{Unconventional superconductivity}



\subsubsection{Spin-fluctuation mediated superconductivity}

\subsection{Hubbard-Kanamori Model}

\subsubsection{Tight Binding Models}

The Tight Binding Model is a central element of condensed matter physics \eqref{TBM}. In this model, electrons are bound in orbitals (called sites) around the lattice ions.
Due to the overlap between the quantum mechanical wavefunctions that describe these sites, electrons are allowed to 'hop' to neighbouring sites. The probability that this hopping process will occur is given by a tunnelling amplitude, which can be calculated using a hopping integral. \par
\medskip
\noindent This work is carried out in the tight-binding model framework, where the magnitude of the tunnelling amplitudes are at first treated as free parameters. 
Here, we build from a simple 2D nearest neighbour hoppping Hubbard model (Section.~\ref{subsec: HubbardModel}) and investigate the effect of introducing and varying the strength of the next-nearest neighbour hopping amplitude (See Fig. \ref{fig:2D Hubbard model}).
We extend these models further by considering the two-orbital per site case. 



\begin{equation} \label{TBM}
    \hat{H}_{TB}(\b{R}) = \sum_{ij\sigma} t_{ij}(\hat{c}_{i\sigma}^{\dagger}\hat{c}_{j \sigma} + h.c)
\end{equation}


\begin{figure}[htbp]  % Placement: Here, Top, Bottom, Page
    \centering
    \includegraphics[width=0.9\textwidth]{2Dhubbardmodel.png}  % Adjust width as needed
    \caption{\textbf{Two-Dimensional Tight-Binding Models:} Three pannels showing the 1NN, 1NNN and 1NN2 models discussed in Section~\ref{subsec:1NNModel},~\ref{subsec:1NNNModel} and~\ref{subsec:1NN2Model} respectively. Fig a) shows the Nearest-neighbour hopping case, where \textit{t} depicts the hopping amplitude between the neighbouring sites. Fig b) shows the inclusion of the next-nearest neighbour hopping, the magnitude given by \textit{t'}.
    Fig c) Shows the extension to the two-orbital case, depicting the same orbital ($t_{orb1}$, $t_{orb2}$) and different orbital ($t_{orb1,2}$) nearest neighbour hopping. Note that this is just a pictorial representation of the orbitals, and that it does not correspond to a particular choice of orbitals or their real space projection. }
    \label{fig:2D Hubbard model}
\end{figure}

\newpage

\subsubsection{Hubbard Model}
\label{subsec: HubbardModel}

The  tight binding model as defined above fails to account for any interactions between neighbouring electrons. This motivates the extension of this model to the Hubbard model \eqref{t Hubbard model}, which includes the (onsite) Coulomb repulsion between electrons. Despite its simple form, this model can describe very rich physical phenomena.
In particular, it becomes very interesting to study when U and t are of comparable order, since it highlights the competing phenomena that takes place in correlated systems. 
The 2D Hubbard model remains unsolved to date, but is able to predict all sorts of correlated phases: it describes metals, insulators, superconductors and other exotic phases (REFERENCE). 
This model has been widely studied since it resembles the structure of the cuprate high-temperature superconductors (REFERENCE). While it has shown to accurately  capture the magnetic and superconducting behaviour that is expected of this family of superconductors, recent findings suggest that the superconducting groundstate 
could indeed be an artefact of some numerical approximation (REFERENCE-ABSENCE...). A large part of this thesis focuses on exploring this toy model and the effect of the nearest-neighbour hopping parameter.



\begin{equation}\label{t Hubbard model}
    \hat{H} = \sum_{ij\sigma} -t_{ij}(\hat{c}_{i\sigma}^{\dagger}\hat{c}_{j \sigma} + h.c) 
    + U \sum_{i} \hat{n}_{i \uparrow} \hat{n}_{i \downarrow}
\end{equation}





\subsubsection{Hubbard-Kanamori Model}
In the case of materials with a multi-band and/or multi-orbital nature, the Hubbard model is not sufficient to capture all of the physical phenomena. This motivates the extension of the Hubbard Model to the Hubbard- Kanamori model by including a Hund's coupling term.
Solving this Hamiltonian is rather challenging, which is why we resort to numerical techniques such as FRG to do so. 

\begin{equation} \label{Hubbard-Kanamori Model}
    H_{int} = U \sum_{is}n_{i,s\uparrow}n_{i,s\downarrow} + \frac{V}{2} \sum_{i,s,t \neq s} n_{is}n_{it} -\frac{J}{2} \sum_{i,s,t \neq s} \vec{S}_{is} \cdot \vec{S}_{it} 
    + \frac{J'}{2} \sum_{i,s,t \neq s} \sum_{\sigma} c_{is\sigma}^{\dagger}c_{is\bar{\sigma}}^{\dagger}c_{it\bar{\sigma}}c_{it\sigma}
\end{equation}

Here, $U$ and $V$ represent the intraorbital interaction of electrons in the same and different orbitals respectively. For generality, the intraorbital exchange $J$ and the 'pair hopping' term $J'$ following from Hund's rule coupling have been separated.  
Note that this Hamiltonian is relevant for the later section of this project, where the model is extended to a two-orbital, two-dimensional Hubbard Model.

\subsection{Theoretical Background in FRG}

FRG falls into the category of many other weak-coupling techniques (such as Mean field-theory, pertubation theory...).
In these theories, the interactions between the electrons are considered to be weak and therefore, one can effectively model the electrons as free particles 
and treat their interactions as a pertubation. In the non-interacting limit, the method is exact. Beyond this limit, it is controlled by the ratio between the interaction
strength and the bandwidth of the system. 

\medskip

\noindent In this section the theoretical framework in which FRG calculations are performed is outlined. The central element of FRG is a flow equation that describes the evolution of the system's action with respect to a scalar parameter Lambda $\Lambda$ \textit{(See section ~\ref{subsubsec:Flow Equation})}.
The flow equation can be solved exactly for a limited number of systems (REFERENCE), so for most sceneraios an approximation has to be made in order to reach a solution in a reasonable computational time. More details of how this approximation is performed and the limitations it presents can be found in Section ~\ref{subsubsec:Truncation scheme}. For the systems explored in this project only two-particle
interactions are considered and any higher order terms are neglected.
This allows for the (effective) action of the system to be separated into three channels, which correspond to the three physical channels: Superconductivity, Spin-Density and Charge-Density Waves.
The calculation is then performed to determine the "winning channel" which will correspond to the respective physical phase that the material exhibits. This rather "hand-wavy" overview of FRG can be viewed as a flowchart in Fig.\ref{fig:FRGflowdiagram}.

\begin{figure}[htbp]  % Placement: Here, Top, Bottom, Page
    \centering
    \includegraphics[width=0.9\textwidth]{FRGflowdiagram.png}  % Adjust width as needed
    \caption{\textbf{FRG Flowchart:} Schematic diagram outlining the TU$^2$FRG calculation steps. Starting from the flow equation and applying the truncation scheme in order to decouple the action into three "physical" terms. 
    The flow equation can then be solved by calculating each channel separatedly and taking the dominating phase to be the channel that diverges.}
    \label{fig:FRGflowdiagram}
\end{figure}

\subsubsection{Flow equation}
\label{subsubsec:Flow Equation}

In this section the derivation of the flow equation is outlined. If interested in further details, the reader is refferred to (REFERENCE). The central elements of statitiscal physics are the partition function, the canonical potential and its Legendre transformations. One can derive all physical observables from these. In quantum many body physics, this partition function is replaced by a partition functional, defined as follows:

\begin{equation}\label{partition functional}
    \mathcal{Z}[\bar{\eta}, \eta] = \int \mathcal{D} \bar{\psi} \mathcal{D}\psi e^{\mathcal{S}[\bar{\psi}, \psi]}e^{(\bar{\eta}, \psi)+(\eta, \bar{\psi})}
\end{equation}

\noindent In the case of fermionic systems, the action in the exponent of Eq.\ref{partition functional} takes the form shown below.
\begin{equation} \label{action}
    \mathcal{S}[\psi, \bar{\psi}] = -(\bar{\psi}, G_0^{-1} \psi) + V[\psi, \bar{\psi}]
\end{equation}

\noindent Here, $V[\psi, \bar{\psi}]$ is an arbitrary many-body interaction and $G_0$ represents the propagator of the noninteracting system. This equation contains the shorthand notation $(...)$, which represents the sum $\sum_x \bar{\psi}(x)(G_0^{-1}\psi)(x), (G_0^{-1}\psi)(x) = \sum_{x’}G_0^{-1}(x,x’)\psi(x’)$. The Grassman field index $x$ represents all the quantum numbers of the single-particle basis and imaginary time.
Note that in the limiting case where $V=0$, the path integral is exactly solveable. The main idea behind FRG is to introduce a cutoff in the non interacting Green's function ($G_0$). 
The cutoff is then interpolated between the solveable intital state and the full path integral solution by susbsequently including electronic correlations. \par

\medskip

\noindent For convinience, the equations are then formulated in terms of the effective action (the Legendre transform of $\mathcal{G}[\eta, \bar{\eta}]$):

\begin{equation} \label{Effective action}
    \mathcal{T}[\psi, \bar{\psi}] = (\bar{\eta},\psi) + (\bar{\psi},\eta) + \mathcal{G}[\eta, \bar{\eta}]
\end{equation}

\noindent where $\mathcal{G}[\eta, \bar{\eta}]$ is given by:

\begin{equation}\label{G term in effective action}
    \mathcal{G}[\eta, \bar{\eta}] = 
    -ln \int{\mathcal{D}\psi \mathcal{D} \bar{\psi}e^{-\mathcal{S}[\psi, \bar{\psi}]}e^{(\bar{\eta}, \psi) +(\bar{\psi}, \eta)}}
\end{equation}

\noindent The next step is to introduce a scalar flow parameter $\lambda$ into the generating functionals defined above. 
This has to be performed such that the generators recover their orginal structure at $\lambda = 0 $.
After a series of algebraic manipulations, which are ommitted here but can be found in (REFERENCE), one arrives at the exact functional flow equation for the effective action:


\begin{equation} \label{eq:ExactFunctionalFlowEquation}
    \frac{d}{d\Lambda} \mathcal{T}^{\Lambda}[\psi, \bar{\psi}] = (\bar{\psi}, \dot{Q}_0^{\Lambda} \psi) - \frac{1}{2} \text{tr} \big( \dot{Q}_0^{\Lambda} (\boldsymbol{\Gamma}^{(2)\Lambda}[\psi, \bar{\psi}])^{-1} \big).
\end{equation}

\noindent Where $\Gamma^{(2)\lambda}[\psi, \bar{\psi}]$ and  $\b{Q}_0^{\Lambda}$ are given by equations (\ref{Gamma term}) and (\ref{Q0 term }) respectively.



\begin{equation}\label{Gamma term}
\Gamma^{(2)\lambda}[\psi, \bar{\psi}] = 
\begin{bmatrix}
\bar{\delta} \delta \Gamma[\psi, \bar{\psi}](x',x) & \bar{\delta} \bar{\delta} \Gamma[\psi, \bar{\psi}](x',x) \\
\delta \delta \Gamma[\psi, \bar{\psi}](x',x)  & \delta \bar{\delta} \Gamma[\psi, \bar{\psi}](x',x)
\end{bmatrix}
\end{equation}



\begin{equation}\label{Q0 term }
\b{Q}_0^{\Lambda} =
\begin{bmatrix}
Q_0^{\Lambda} & 0 \\
0 & -Q_0^{\Lambda t}
\end{bmatrix}
= diag(Q_0^{\Lambda}, - Q_0^{\Lambda t}),
\end{equation}


\subsubsection{Truncation scheme}
\label{subsubsec:Truncation scheme}
\subsubsection{Decoupling of flow equation}


After constraining ourselves to the case of two-particle interactions in the framework of translationally invariant systems, one can decouple  the evolution of the two-particle coupling as a function of the flow parameter $\lambda$ into three channels:

\begin{equation}\label{3 channels}
    \frac{d}{d\lambda} V^{\lambda}(k1, k2, k3) = \mathcal{T}_{pp}^{\lambda}(k1,k2,k3) + \mathcal{T}_{cr-ph}^{\lambda}(k1,k2,k3) +\mathcal{T}_{d-ph}^{\lambda}(k1,k2,k3)
\end{equation}

\noindent Here, the three channels correspond to a particle-particle, crossed particle-hole and (three) direct particle-hole terms given explicitly below and they represent all possible ways in which the two particle interactions can occur in our correlated system. When applied to 2D systems, the solution to the flow equation is usually inacecessible. For this matter, there exist several numerical approaches of which this TUFRG method remains as the most powerful one.

\subsubsection{Instability calculation}




\section{Computational Methods}

\subsection{Diverge}

\subsection{Convergence of Calculation}
\label{subsec:convergence}

\subsubsection{Form factor convergence}

\subsubsection{Number of k points convergence }

\subsection{Calculation of Susceptibilities}

\subsubsection{Nesting vectors}

\subsubsection{Superconducting order parameters}

\section{Results and discussion}

\subsection{1NN Model}
\label{subsec:1NNModel}

We define the 1NN model as the 2D Hubbard model with a single orbital per site,  allowing hopping between nearest-neighbour sites \textit{(a diagramatic representation can be found in Fig.\ref{fig:2D Hubbard model}.a)}. 
As outlined in Section \ref{subsec: HubbardModel}, solving this model near half-filling becomes a hard endeavour. 
This section explores the solution to the 1NN Model Hamiltonian using  two-particle-interaction-truncated FRG.
Here, a phase diagram in terms of the On-site Coulumb repulsion $U$ and chemical
potential $\mu$ is presented and discussed in detail. The calculations were carried out using a 20x5 $n_k$x$n_{kf}$ grid and a form factor
of 4\AA, with a  nearest-neighbour hopping parameter of 1eV.  The choice of convergence parameters was determined by the methods discussed in Section \ref{subsec:convergence}. 
Results are presented for Coulumb repulsion values  ranging from 1-20 eV and chemical potential
values spanning the entire energetic bandwith of the model (from -4eV to 4eV). 
Whilst there exists previous work on the 2D Hubbard Model using FRG (REFERENCE), most of it focuses on small areas of the phase diagram and none of it explores the effect of varying the 
onsite Coulumb repulsion. Therefore, the work presented here covers a much wider range of parameters than what has previously been explored. 
\medskip




\noindent In real materials,  the chemical potential is an easily tuneable parameter due to its strong connection to the electronic doping of the system.
However, controlling the magnitude of the Coulumb repulsion between electrons is not that straightforward. 
Moreover, some of the interesting features discussed in this report lie outside of the physical regime. (ie- in our FRG calculation a weak-coupling limit was 
assumed and therefore any Coulumb repulsion values above the magnitude of the  bandwidth of the material (8eV) can be considered to be unphysical in this limit.)
Therefore, the analysis performed in this project does not neccesarily provide a route to enhance the superconductivity in real materials that resemble the models investigated. Instead, the main aim is to understand how varying these parameters affects the correlated phases observed in the model. It is also important to note, as discussed in Section \ref{subsubsec:Truncation scheme}, that any analysis on the transition temperature
of the superconducting regions must be treated qualitatively and that this work does not claim to have found Superconducting regions at temperatures of the order of thousands of Kelvin in any of the models discussed. \par 

\medskip
\noindent The complete phase diagram for the values discussed above is shown in Fig.\ref{fig:1NNpd}. This particle-hole symmetric phase diagram shows a magnetic dome sandwiched between two superconducting regions. Moreover, narrow magnetic stirpes appear for even
integer chemical potential values.
\par

\medskip

\noindent Paragraph comparing results to the Cuprates!





\begin{figure}[htbp]  % Placement: Here, Top, Bottom, Page
    \centering
    \includegraphics[width=1.25\textwidth]{1NNphasediagram.png}  % Adjust width as needed
    \caption{\textbf{Phase diagram for the 1NN model}:  (t =1eV, nkxnkf = 20x5, ff = 4\AA) as a function of On-site Coulumb Repulsion $U$ and chemical potential $\mu$. 
    Figure shows the four phases observed in the 1NN model: SC (Superconductivity), SDW (Spin-Density Wave), CDW(Charge Density Wave) and FL (Fermu-Liquid).
    Calculated points in the phase diagram are showed by the 'x' markers and a lighter-couloured background is used to depict interpolated regions between these points. }
    \label{fig:1NNpd}
\end{figure}

\newpage


\subsubsection{Superconductivity in the 1NN Model}

\begin{figure}[htbp]  % Placement: Here, Top, Bottom, Page
    \centering
    \includegraphics[width=1.0\textwidth]{1NNSC.png}  % Adjust width as needed
    \caption{\textbf{Superconductivity in the 1NN model}:  
    Fig a) Superconducting order parameter projected on Fermi Surface at $\mu$ = 1.00 eV, 
    for U =10.00eV. The order parameter is antisymmetric about a 90 degree rotation and hence
    it exhibits d-wave symmetry. 
    Fig b) Transition temperature (Tc) as a function of chemical potential ($\mu$) for U =10.00eV.
    Tc is enhanced closest to the magnetic instability.   
    }
    \label{fig:1NNSC}
\end{figure}



\subsubsection{Magnetic stripes in the 1NN Model}

\begin{figure}[htbp]  % Placement: Here, Top, Bottom, Page
    \centering
    \includegraphics[width=0.9\textwidth]{1NNstripes.png}  % Adjust width as needed
    \caption{\textbf{Magentic stripe in the 1NN model for doping at $\frac{1}{8}$ of the bandwith}: Fig a) Transition temperature as a function of Coulumb repulsion (U) along the Magnetic stripe at $\mu$ =1eV. 
       Fig b) Magnetic susceptibility along high symmetry path for U =2.00 eV. 
       Fig c) Plot of the Superconducting order parameter projected on top of the Fermi-surface of the 1NN model for U = 10.00eV and $\mu$ =1.00eV.
       Fig d) Magenetic susceptibility along high symmetry path for U= 18.00eV. Plots showing the susceptibility as a function of \b{q} for both magnetic regions. The Ferromagnetic SDW is supressed by a superconducting phase at U$\approx$ 6.00eV. At larger values of U,  the SDW phase is recovered but with an Anti-Ferromagnetic ordering instead.  }
    \label{fig:1NN_stripes}
\end{figure}


\subsection{Effect of next-nearest neighbour hopping (1NNN model)}
\label{subsec:1NNNModel}


\begin{figure}[htbp]  % Placement: Here, Top, Bottom, Page
    \centering
    \includegraphics[width=1.0\textwidth]{1NNN.png}  % Adjust width as needed
    \caption{\textbf{1NNN model:} a) Phase diagram for the 1NNN model as a function of Coulumb repulsion U and chemical potential $\mu$ (t =1eV, nkxnkf = 20x5, ff = 4\AA) for t'=0.00, 0.25, 0.50, 0.75 eV. 
    b) Transition temperature for the superconducting region.
    c) Single plot of the band structure along high-symmetry path for corresponding values of t'.
    d) Single plot of the Fermi surface for all values of t' considered.}
    
    \label{fig:1NNN}
\end{figure}

\newpage

\subsubsection{Superconductivity in the 1NNN Model}

Important to note that results shown in Fig \ref{fig:1NNNSC} are the plots for the case of U = 9.00eV and although this
value is unphysical since we constrain ourselves to the weak-coupling approximation in which we assume that U is of the orders of
magnitude of the bandwidth of the material. 


\begin{figure}[htbp]  % Placement: Here, Top, Bottom, Page
    \centering
    \includegraphics[width=0.90\textwidth]{1NNNSC_075.png}  % Adjust width as needed
    \caption{\textbf{Change in the superconducting order parameter in the t' =0.75eV 1NNN model}: Fig a) Critical temperature
    as a function of chemical potential $\mu$ for u = 9.00eV.
    The lower panel shows the magnitude of the superconducting order parameter plotted on top of the fermi surface for U= 9.00eV and $\mu$ = -2.6eV, -1.8eV and -1.4eV. These are shown in Figs b)
    c) and d) respectively.  }
    \label{fig:1NNNSC}
\end{figure}




\subsubsection{Magnetic stripes in the 1NNN model}

\begin{table}[h]
    \centering
    \begin{tabular}{|c|c|c|c|c|c|c|}
        \hline
       t'(eV) & $\mu$ (eV) & Competing with other phase?& Nesting vector & Magnetic ordering  \\
        \hline
        0.00 & 1.00 & Yes (SC) & (0,0)-($\pi$, $\pi$)&  Changing from FM-AFM\\
        \hline
        0.00 & 2.00 &  No  & (0,0)-(0, $\pi$)  & Changing from FM to Commensurate\\
        \hline
        0.25 & 2.40 &  No  & (0,0)  & FM \\
        \hline
        0.25 & 1.40 &  Yes (CDW)  & (0,0)  & FM \\
        \hline
        0.25 & -2.00 & No  & (0,0)-($\pi$, $\pi$)  & FM - AFM\\
        \hline
        0.25 & 0.60 &  No  & (0,0)  & FM \\
          
        \hline
    \end{tabular}
    \caption{\textbf{Survey of stripes in the 1NNN model.} Table summarising the location and magnetic ordering of the Stripes in the 1NNN model .}
    \label{tab:StripesSummary}
\end{table}

\subsubsection{Continous variation of next-nearest neighbour hopping}

\subsection{Effect of bi-orbital system (1NN2 model)}
\label{subsec:1NN2Model}

\begin{table}[h]
    \centering
    \begin{tabular}{|c|c|c|c|c|c|c|}
        \hline
       Model &$t^{[1,0,0]}_{3z^2-r^2}$  &$t^{[0,1,0]}_{3z^2-r^2}$  &  $t_{x^2 - y^2}$ &  $t^{[1,0,0]}_{x^2 - y^2 -3z^2-r^2} $ & $t^{[0,1,0]}_{x^2 - y^2 -3z^2-r^2} $ & $t_{\perp}$ \\
        \hline
        & -0.781 & -0.719  &  -0.375 & -0.402 & -0.310 & -2.50\\
        \hline
        1NN2MN & on &  on  & on  & off & off & off \\
        \hline
        1NN2MY & on &  on  & on  & on & on & off \\
          
        \hline
    \end{tabular}
    \caption{\textbf{Nearest neighbour hopping parameters for 2D  two-orbital Hubbard models.} First row shows the calculated hopping parameters, note that  $t_{\perp}$ corresponds to the intralayer hopping between the $d_{3z^2-r^2}$ orbitals. Other rows show which hopping parameters were included in each of the two models.}
    \label{tab:2D2orbparams }
\end{table}



\subsubsection{Superconductivity in the 1NN2 model}

\subsubsection{Effect of orbital hybridisation in the 1NN2 model}



\section{Conclusion and Outlook}


\newpage

\bibliographystyle{unsrt}  % Choose a style such as plain, alpha, abbrv, etc.
\bibliography{references}

\end{document}