\documentclass{article}
\usepackage{graphicx} % Required for inserting images

\title{Ideas -Masters Project}
\author{Maria Ramírez Rodríguez}
\date{February-May 2025}

\begin{document}

\maketitle

\section{1NN and 2NN models}

\subsection{Overview}

The 1NN model describes the 2D Hubbard model, only considering nearest neighbour hopping.
The 2NN model describes the exact same physical system, but the unit cells are now considered to be 2 atom units cells. 
When performing the FRG calculation of both models, one expects to see the same phase diagram output.

Several things have to be taken into account: k convergence and form factor convergence.
Currently, the parameters nk and nkf have been set to be 20 and 5 respectively. An interesting test would be to thoroughly check whether there are lower values of nk and nkf for which the calculation returns the correct ground state.
The form factor convergence is currently being tested for a range of parameters in units of \AA.



\subsection{Form factor convergence}

Form factor convergence in the 2NN model is being tested for values of 1-10 \AA. This is being tested for a 2NN Hubbard model with a coulumb repulsion U = 3 and $\mu$ = 0.2 eV.
The nk and nkf parameters used are 20 and 5 respectively. 

For these particular parameters, the calculation seems to converge (Tc is stabilised) at a value of around 3-4 \AA in the 2NN model. The 1NN model behaves slightly different, where the convergence for the 


\subsection{K-points convergence}


\end{document}